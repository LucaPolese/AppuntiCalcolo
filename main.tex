\documentclass[12pt,a4paper,headings=optiontohead,openany]{scrbook}
\usepackage[utf8]{inputenc}
\usepackage[italian]{babel}
\usepackage[margin=1.8cm,bottom=7em]{geometry}
\usepackage[subpreambles=false]{standalone}
\usepackage{algorithm}
\usepackage{algorithmic}
\usepackage{amsmath}
\usepackage{amssymb}
\usepackage{amsthm} 
\usepackage{cancel}
\usepackage{enumitem}
\usepackage{float}
\usepackage{graphicx}
\usepackage{hyperref}
\usepackage{mathtools}
\usepackage{xcolor}

\usepackage[normalem]{ulem}

\usepackage{blkarray}% http://ctan.org/pkg/blkarray

\newcommand{\matindex}[1]{\mbox{\scriptsize#1}}% Matrix index

\renewcommand{\qedsymbol}{\rule{0.7em}{0.7em}}
\DeclarePairedDelimiter{\abs}{\lvert}{\rvert}
\newcommand{\inter}{\begin{matrix}\prod\end{matrix}}
\newcommand{\verteq}{\rotatebox{90}{$\,=$}}
\newcommand{\equalto}[2]{\underset{\scriptstyle\overset{\mkern4mu\verteq}{#2}}{#1}}
\DeclarePairedDelimiter{\norma}{\lVert}{\rVert}
\newtheorem*{esempio}{Esempio}

\hypersetup{
    colorlinks,
    citecolor=black,
    filecolor=black,
    linkcolor=black,
    urlcolor=blue
}

\DeclareMathSizes{12}{13}{10}{9}

\begin{document}

%------------------------------------------------------------------------------------------------------
%------------------------------------------------------------------------------------------------------
%-----------------------------------------------INTESTAZIONE-------------------------------------------
%------------------------------------------------------------------------------------------------------
%------------------------------------------------------------------------------------------------------

\begin{titlepage}

\newcommand{\HRule}{\rule{\linewidth}{0.5mm}} % Defines a new command for the horizontal lines, change thickness here

\center % Center everything on the page
 
%----------------------------------------------------------------------------------------
%	HEADING SECTIONS
%----------------------------------------------------------------------------------------


\large Dalle dispense del\\[0.5cm] % Major heading such as course name
\textsc{\Large Prof. Marco Vianello}\\[1.5cm] % Minor heading such as course title

\includegraphics[scale=0.13]{maths.png}\\[0.3cm]
%----------------------------------------------------------------------------------------
%	TITLE SECTION
%----------------------------------------------------------------------------------------

\HRule \\[0.4cm]
{ \huge \bfseries Appunti di}\\
{ \huge \bfseries Calcolo Numerico\\[0.15 cm]} % Title of your document
\HRule \\[1cm]
 
%----------------------------------------------------------------------------------------
%	AUTHOR SECTION
%----------------------------------------------------------------------------------------

\begin{minipage}{0.4\textwidth}
\begin{flushleft} \large
\emph{\Large{Autori:}}\\
2019-2020\\
\quad \quad Michele \textsc{Veronesi}\\
\quad \quad Silvia \textsc{Bazzeato}\\
\quad \quad Simone \textsc{De Renzis}\\
\quad \quad Michele \textsc{Baldisseri}\\
\quad \quad Andrea \textsc{Cecchin}\\
\quad \quad Stefano \textsc{Rizzo}\\
\quad \quad Leonardo \textsc{Tredese}\\
\quad \quad Alessandro \textsc{Pirolo}\\
\quad \quad Filippo \textsc{Pinton}\\
2020-2021\\
\quad \quad Luca \textsc{Polese}\\
\end{flushleft}

\end{minipage}\\[2cm]

%----------------------------------------------------------------------------------------
%	DATE SECTION
%----------------------------------------------------------------------------------------

\LARGE Università degli Studi di Padova\\[0.4cm]
\textsc{\large Dipartimento di Matematica}\\[0.05cm]
\textsc{\large Corso di Laurea in Informatica}\\[1cm] 
{\Large Anno accademico 2020 - 2021}\\[1cm] 

\vfill % Fill the rest of the page with whitespace

\end{titlepage}


\begin{center}
\pagebreak

\section*{Premessa}
\begin{minipage}{0.9\textwidth} \large

Dagli appunti raccolti durante l'anno accademico 2019-2020 costellato di difficoltà, è nata l'esigenza di unificare il dettagliato materiale in un documento quanto più possibile uniforme e fruibile. Chi dedicatosi a una cosa e chi ad un'altra, la trascrizione è frutto della collaborazione di studenti. \\

Si pone all'attenzione degli studenti che la seguente versione è stata aggiornata e corretta nell'anno 2020-2021 in seguito alle lezioni tenute in diretta dal Prof.Vianello.\\
La versione originale del documento è fruibile a questo \href{https://github.com/micheleveronesi/AppuntiCalcolo}{indirizzo}\\

Non ci si aspetti una forma perfetta e si studi (come sempre del resto) con occhio critico e attento agli errori.\\

\end{minipage}
\end{center}
\pagebreak
\tableofcontents
% ############ CAPITOLO 1 ##################
\chapter{Sistema floating-point e propagazione degli errori; costo computazionale}
%\chaptermark{Floating-point, errori, costo computazionale}
\input{tex/lezione1}
\input{tex/lezione2}
\input{tex/lezione3}
\input{tex/lezione4}
\input{tex/lezione5}
\input{tex/lezione6}
\input{tex/lezione7}

% ############# CAPITOLO 2 #####################
\chapter{Soluzione numerica di equazioni non lineari}
\input{tex/lezione8}
\input{tex/lezione9}
\input{tex/lezione10}
\input{tex/lezione11}

% ############ CAPITOLO 3 ######################
\chapter{Interpolazione e approssimazione di dati e funzioni}
\input{tex/lezione12}
\input{tex/lezione13}
\input{tex/lezione14}
\input{tex/lezione15}

% ############## CAPITOLO 4 #####################
\chapter{Integrazione numerica e derivazione numerica}
\input{tex/lezione16}
\input{tex/lezione17}
\input{tex/lezione18}

% ################ CAPITOLO 5 #####################
\chapter{Elementi di algebra lineare numerica}
\input{tex/lezione19}
\input{tex/lezione20}
\input{tex/lezione21}
\input{tex/lezione22}
\input{tex/lezione23}
\input{tex/lezione24}

\end{document}
